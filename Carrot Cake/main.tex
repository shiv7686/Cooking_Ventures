\documentclass[10pt,landscape]{article}
\usepackage{graphicx}
\usepackage{amssymb,amsmath,amsthm,amsfonts}
\usepackage{multicol,multirow}
\usepackage{calc}
\usepackage{ifthen}
\usepackage[landscape]{geometry}
\usepackage[colorlinks=true,citecolor=blue,linkcolor=blue]{hyperref}


\ifthenelse{\lengthtest { \paperwidth = 11in}}
    { \geometry{top=.5in,left=.5in,right=.5in,bottom=.5in} }
	{\ifthenelse{ \lengthtest{ \paperwidth = 297mm}}
		{\geometry{top=1cm,left=1cm,right=1cm,bottom=1cm} }
		{\geometry{top=1cm,left=1cm,right=1cm,bottom=1cm} }
	}
\pagestyle{empty}
\makeatletter
\renewcommand{\section}{\@startsection{section}{1}{0mm}%
                                {-1ex plus -.5ex minus -.2ex}%
                                {0.5ex plus .2ex}%x
                                {\normalfont\large\bfseries}}
\renewcommand{\subsection}{\@startsection{subsection}{2}{0mm}%
                                {-1explus -.5ex minus -.2ex}%
                                {0.5ex plus .2ex}%
                                {\normalfont\normalsize\bfseries}}
\renewcommand{\subsubsection}{\@startsection{subsubsection}{3}{0mm}%
                                {-1ex plus -.5ex minus -.2ex}%
                                {1ex plus .2ex}%
                                {\normalfont\small\bfseries}}
\makeatother
\setcounter{secnumdepth}{0}
\setlength{\parindent}{0pt}
\setlength{\parskip}{0pt plus 0.5ex}
% -----------------------------------------------------------------------

\title{Carrot Cake}

\begin{document}

\raggedright
\footnotesize

\begin{center}
     \Large{\textbf{Carrot Cake}} \\
     \small{\today}
\end{center}
\begin{multicols}{3}


\section{Notes}
\begin{tabular}{l l} 
 Yields & 6 servings \\ 
 Prep Time & 30 minutes \\ 
 Chill Time & 1 hour \\ 
 Cook Time & 25 minutes \\
 Bake Temperature & 350\textdegree F \\
 Utensil Type & Metal
\end{tabular}

\bigskip

\section{Ingredients}
\begin{tabular}{l l} 
 All Purpose Flour & 280g \\ 
 Baking Soda & 2 tsp \\
 White Sugar & 100g \\
 Brown Sugar & 150g \\
 Ginger & 1 tsp  \\ 
 Cinnamon & 2 tsp  \\
 Nutmeg Grated & 1/2 tsp  \\ 
 Salt & 1 tsp  \\ 
 Oil & 1 cup \\
 Whole Egg & 4 \\
 Vanilla Extract & 2 tsp \\
 Carrots Shredded & 300g \\
 Walnuts & 1 cup \\
 Raisins & 1/3 cup \\
 Heavy Cream & Varies \\
 Cream Cheese & 8 oz \\
\end{tabular}

\vfill\null
\columnbreak

\section{Instructions}
\begin{enumerate}
    \item Whisk the flour, baking soda, brown/white sugar, ginger, cinnamon, nutmeg and salt together in a large bowl. Set aside.
    \item Whisk the eggs and oil together until a uniform texture is formed. Finally, whisk in the vanilla extract.
    \item Pour the wet ingredients into the dry ingredients and mix together
    \item Add some heavy cream to get the batter to a cake batter like texture
    \item Add raisins, walnuts and carrots and mix well together.
    \item Pour the batter to a cake pan and bake for 25 - 30 minutes at 350\textdegree F
    \item Take the cake out of the pan and let it cool down completely.
    \item For the frosting, take cream cheese and mix it with sugar and heavy cream.
    \item After the cake has completely cooled down, apply the frosting on the top.
    \item Place it in the fridge for 1 hr for the frosting to set.
    
\end{enumerate}
\bigskip
CAUTION: All ingredients must be room temperature
\vfill\null
\columnbreak
 
\section{Results}
\begin{itemize}
    \item Cake came out pretty well. The crushed walnuts and raisins definitely added flavor to every bite. 

\end{itemize}
\bigskip
\section{Try out next batch}
\begin{itemize}
    \item Add more frosting for middle layer
    \item Wait for cake to completely cool down or else frosting will melt
\end{itemize}
\vfill\null
\columnbreak
\end{multicols}

\bigskip
Inspired from Joshua's carrot cake recipe https://www.youtube.com/watch?v=cVO7WOlJFY8
\end{document}
